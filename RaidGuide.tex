\documentclass[12pt]{amsart}

% For diagrams If I need to add them later
\usepackage{geometry}

% Change to A4 UK size
\geometry{a4paper}

% Title
\title{Raid Guide}

% Author
\author{Night Sentinels, Dragonmaw (EU)}

% start the document
\begin{document}

% Add the title page
\maketitle

\newpage

\tableofcontents

\newpage

\section{Introduction}
The idea behind this project is to put together all the information that is needed for anyone who would like to start raiding. It will focus initially on the current working operations of the guild and expectations. It then goes on to outline gear requirements along with advice on getting better gear and hopefully increased performance. Other aspects like addons, consumables and vent are also covered. Finally boss tactics to all current raid bosses are outlined.\\

I hope that this makes it easier for anyone who has just hit their level cap and is thinking about raiding or has maybe just joined us. The plan is for this guide to continually evolve and be kept up to date. It also means that there will be a common and consistent set of procedures so that raids can run as smoothly as possible.\\

Have fun!\\

\section{Raid Rules and Procedures}
\subsection{Introduction}
For anyone that has raid experience and maybe even some that have been on only a few raids, one thing that you will notice quickly is that different raid leaders/guilds do things differently. This section aims to inform you how things are done at <Night Sentinels>, what you can expect from us as raid leaders and in turn what is expected from you as a raider.\\

One thing I would like to make clear at the start is that raiding is very different from other aspects of the game. When you raid you are interacting and supporting a group of other people. Your behavior, actions and performance have a direct impact on the other members of your group. If you turn up late for a raid late, under-geared or without the knowledge needed you not only waste your own time, but you waste others too. The aim is to have everyone perform as well as they can so that everyone can have fun.\\

\subsection{Raid Times and Invites}
Raids will be scheduled from the in game calender by a raid leader and invites sent from there. Usually we raid at 20:30 Server Time (ST) on Sundays and 22:00 ST on Thursdays. It is common though for raids to also be scheduled for Tuesdays at 22:00 ST and Saturday at 22:00 ST.  The event in the calender will let you know which raid it is, how long the raid is intended for and which bosses we aim to do. This gives you the opportunity to look up tactics if needed.\\

When you see an invite please respond as soon as you can as it makes setting up much easier. If you know for certain that you are available at the raid time and would like to attend you can accept the invitation. If however you are unsure as to whether you are available then you should decline the invitation, and if you become available later you can accept.\\

Thirty minutes before raid time the raid leader will start to put the raid together. He/She will make a group from members who accepted AND are online. Please make an effort to be online at least thirty minutes before a raid starts. Raid invites will then be given to members chosen for that raid. If you are not chosen for a raid do not be disappointed. Some raids are often over subscribed and raid leaders will do all they can to include all that want to raid, but raids do have limited slots.\\

\subsection{Loot}
At Night Sentinels we do not operate any type of DKP system or similar. Currently if you have contributed to a boss kill you are eligible to a roll at the loot. This is with some exceptions though. Priority goes to main spec (your main spec is considered the spec you are currently playing in that raid). If no raid members need the item for their main spec then it will be offered for off spec.\\

After the boss kill the loot will be linked by the Master Looter, to show what has been dropped (as of patch 3.1 there is no reason for anyone other than the Master Looter to loot a boss kill, as emblems are distributed automatically). Important to note, no raid member is to loot until all members of the raid are alive and in the room. The Master Looter will then link the first item. If you would like the item you should print your name in raid chat. The Master Looter will then confirm who is allowed to roll on an item and then ask those chosen to roll. The member with the highest roll will be given the loot.\\

There is no limit to the number of pieces of loot may be obtained in a raid run, but you are allowed to pass on items if you would like and generosity does not go unnoticed. If a piece of loot is merely a 'side-grade' where by it would be a large upgrade to another member of the raid it is common courtesy (although not mandatory) to pass on the item.\\

\subsection{Raid Leadership}
Raids are lead by one or two raid leaders who will decide which bosses to attempt and which tactics to use. It is imperative to listen to them carefully as you may be given a specific task for a certain boss fight. There may be times when a raid leader will ask for advice or extra information from the raid group, this should be the only time in which raid members give opinions or comments on tactics. If you would like to raise a point it is considered polite to write it in raid chat rather than to interrupt the raid leader over Vent (Ventrilo to be covered later). Finally, Raid Leaders put a large amount of effort in preparing for raids, reading up on tactics and trying to make sure everyone has fun. They always appreciate a thank you.\\

\section{Gear}
\subsection{Introduction}
One of the most common questions asked by new guild members and those who have just reached the level cap, is regarding the gear requirement for raids. In this section I will outline what gear you will be expected to bring to be able to perform adequately in a raid environment. A key point to make though is that gear is just one of the many factors which contribute to a good raider. Having good gear does not guarantee you a place in a raid and visa versa having poor gear does not rule you out of raiding. I frequently see players perform better than another player of the same class with far superior gear. You should see gear as one of the ways of making you a better raider, not the only way. Also with the advent of Wrath of the Lich King, early raid content is more accessible than ever. Getting enough gear to be able to raid is no longer a daunting task.\\

\subsection{Requirements}
To be eligible to come to Naxxramas and Obsidian Sanctum the minimum gear requirement is ilv187 with all pieces. The 'ilvl' can be seen on the tooltip of items and gives a guide to its quality. You should definitely be looking at the ilevel of items rather than their rarity when deciding on gear. Of course the quality of gear is irrelevant if it is the wrong gear; choosing the right gear is just as important. We expect all members wishing to raid to have at least a general understanding of their class's (and spec's) stat priority and weighting. If you are unsure which gear type is best for you simply ask in guild chat or spend a few minutes reading a few forums online. There are also add-ons (covered later) which can give you detailed information to gains and losses of equipping gear.\\

We would also expect gear to be enchanted and gemmed appropriately. It is understandable to have limited enchants and gems on gear that is likely to be upgraded soon, so lower end enchants and cheap gems would not be a problem. Once you have obtained pieces that are unlikely to be replaced you should have them enchanted with the best enchants and gems available to you. The guild have Enchanters and Jewelcrafters who can help you with getting your gear enhanced. There is also the ability to purchase cheap mats from the guild bank, although this is not available to new recruits.\\

Obtaining gear for raiding is extremely simple. I would suggest starting to run Heroic level instances as soon as possible, as gear from there is equivalent to that from early tier raiding. A common method for raiders is to check online (www.wowhead.com is a good start) for which heroic instances drop their ideal gear and then to create a small wish list. You can then try and run Heroics on your wishlist to try and obtain your gear. As with The Burning Crusade, WotLK has a badge system. All Heroic level instances and normal level raid bosses drop Emblems of Heroism which can be exchanged for pieces of gear. It is advisable to save up and get the Tier 7 pieces first as these will be your most solid upgrades and also give helpful set bonuses\\

Finally all classes have access to crafted gear of very good quality. There are Epic ilvl200 crafted pieces available for each class as well as Rare ilvl187 pieces. The ilv187 pieces, although often slightly more tuned towards PvP are a good starting option and require very minimal mats. The Epic items require a little more mats but are solid items and will carry you a good way through early raids. There are maximum level crafters for all professions in the guild, with most patterns, who will be more than willing to help you get some gear crafted.\\ 

\subsubsection{Tanks}
Tanks will need to have 540 Defense before they can step into a raid. Having 539 is simply not good enough. Being below the Defense threshold (nb: This is a threshold NOT a 'cap' as with other stats, increasing Defense is still advantageous although suffers from diminishing returns past the threshold) gives bosses the ability to critically strike you. Certain bosses could easily critically strike greater than you entire health pool, so getting to 540 Defense should be your number one priority.\\

After obtaining 540 Defense depending on your class you will be aiming to increase your avoidance, mitigation and health pool. To raid we would expect to see your HP approaching 30k when fully raid buffed.

\subsubsection{DPS}
The majority of raid bosses in WotLK have an enrage timer, meaning the boss needs to be downed within a set duration or else the raid will wipe. It is necessary then to set some minimum standards of performance. We would expect to see DPS of at least 1.5k for early raiding, this can be found be using an add-on like Recount (discussed later) and running instances or using training dummies in the major cities.\\

To enable the best output from your class you will need to have the required hit rating or spell hit rating . Further details on hit rating can be be found at (www.wowwiki.com). Raid bosses have an increased chance to miss and so your gear will need to reflect this, to enable you not to miss and therefore have the greatest damage possible. Most classes have talents which can help increase your chance to hit, and so these are often recommended if you are under you hit 'cap'. The 'cap' refers to the amount of hit rating needed from gear, talents and buffs that enable you to never miss a raid boss (apart from built in miss rates for melee classes). Specific hit caps are different for casters and melee. For melee and hunters the hit cap is 263 Hit rating equivalent to 8\% with Spell Casters having a larger cap [TBC].\\

\subsubsection{Healers}
It is hard to give specific requirements for healers due to the unquantifiable nature of measuring healing. If you are within the overall gear requirements for a raid you should have no problems helping out with healing, especially if working with well geared tanks.\\

\section{UI, Add-ons and Vent}
There are a wide range of tools which help us with raiding. This sections will look at the tools which are used by the guild and which you will be expected to have.\\
\subsection{Vent}
Ventrillo is a program which runs on your computer (completely independently from WoW) and enables all raid members to chat with each other and listen to the raid leader. It cannot be overstated how important it is to be able to listen into the raid and ideally communicate back to the raid. You can obtain Ventrilo from (www.ventrilo.com) and is available for both Windows, Mac OS X and *nix. We currently run from a 3.0.5 Server so you will need to download the 3.0.5 Client. Once downloaded install the application and run it. There are a large number of setup options in Vent, you can safely ignore most of them. The first you will need is the log in details for our server:\\

IP:	vent16.gameservers.com

Port: 4747

Pass: Please ask a guild officer for the current password\\

Please note that the details of our server are not to be given out to anyone without express permission from an Officer or Guild Leader. Once the details have been entered correctly you should be able to connect to the server. You can then set up a Push To Talk (PTT) button if you wish, which will allow you to speak to the raid at the push of a button. If you need further information on configuring Vent please ask in guild chat, ideally not a few minutes before raid time.\\

\subsection{UI}
UI refers to your User Interface within the game. When raiding you should be expected to have some form of raid frames which enable you to see all the members of the raid. It is imperative that healers have a good setup which enables then to heal any member of the raid and see which are out of range. The majority of spells and buffs are now raid wide and as such raid groups are not particularly necessary. Common healing raid frames are Grid and Healbot among others. It does not matter though which raid frames you use, as long you have them set up\\

Your UI should also allow you to see all debuffs and buffs as in many fights depend on knowing which debuffs you have. In addition some fights require the use of pet and vehicle bars, these should also work with your current UI set up.\\

\subsection{Add-ons}

\subsubsection{Recount}
This add-on records the damage and healing of all members in the raid. It is the most common add-on for measuring players DPS and so all DPS class would be expected to have Recount installed in order to monitor their own and others performance.\\

\subsubsection{Deadly Boss Mods}
This add-on is the preferred mod for help with raid bosses. It gives warnings of upcoming boss abilities, incoming phases and enrage timers. It is extremely helpful and recommended for all raiders\\

\subsubsection{Rating Buster}
This displays helpful extra information on tooltips of gear. It will show the advantages and disadvantages of equipping a piece of gear relative to the currently equipped item. This helps a lot when looking for upgrades to your gear.

\section{Consumables}
When raiding you will be expected to be as buffed as possible for you to get the best out of your gear and spec. You will be expected to bring flasks to all raids, or alternatively a combination of elixirs. These flasks although costly can make the difference between a wipe at 1\% and a boss kill. We have number of fully leveled Herbalists and Alchemists in the guild which can help you obtain mats and make flasks or elixirs\\

Food buffs in WotLK have received a very large increase in potency. All raid members are also expected to bring their own buff food. Again we have fully leveled chefs in the guild as well as fisherman that can help you obtain mats. It is not uncommon though for raid members to drop a 'feast' for the whole raid to use. You should not expect this and come prepared with your own food.\\

Potions are also very helpful in the raid environment. Healing, mana and other potions should also be brought depending on class and used when needed\\

\section{General Points}
There are some other general expectations of you for raids. You should be expected to be prompt and attentive for raids, and not assume that you will be summoned. You should be fully repaired, stocked with reagents as well as ammo if needed. You are to be courteous and empathic towards fellow guild and raid members at all times and give 100\% at all times
\section{Boss Tactics}
\subsection{Introduction}
To be added later and will most likely incorporate the previously written guide by Pokee
\subsection{Naxx 10-Man}
\subsubsection{Introduction}
\subsubsection{Trash}
\subsubsection{Anub�Rekhan}
\subsubsection{Grand Widow Faerlina}
\subsubsection{Maexxna}
\subsubsection{Instructor Razuvious}
\subsubsection{Gothik the Harvester}
\subsubsection{The Four Horsemen}
\subsubsection{Noth the Plaguebringer}
\subsubsection{Heigan the Unclean}
\subsubsection{Loatheb}
\subsubsection{Patchwerk}
\subsubsection{Grobbulus}
\subsubsection{Gluth}
\subsubsection{Thaddius}
\subsubsection{Sapphiron}
\subsubsection{Kel�Thuzad}

\subsection{Obsidian Sanctum 10-Man}
\subsubsection{Introduction}
\subsubsection{Trash}
\subsubsection{Zero Drakes Up}
\subsubsection{One Drake Up}
\subsubsection{Two Drakes Up}
\subsubsection{Three Drake Up}

\subsection{Eye of Eternity 10-Man}
\subsubsection{Introduction}
\subsubsection{Phase One}
\subsubsection{Phase Two}
\subsubsection{Phase Three}

\subsection{Ulduar}

\end{document}